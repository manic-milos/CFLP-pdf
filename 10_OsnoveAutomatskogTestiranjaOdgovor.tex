 % !TEX encoding = UTF-8 Unicode

\documentclass[a4paper]{report}

\usepackage[T2A]{fontenc} % enable Cyrillic fonts
\usepackage[utf8]{inputenc} % make weird characters work
\usepackage[serbian]{babel}
%\usepackage[english,serbianc]{babel}
\usepackage{amssymb}

\usepackage{color}
\usepackage{url}
\usepackage[unicode]{hyperref}
\hypersetup{colorlinks,citecolor=green,filecolor=green,linkcolor=blue,urlcolor=blue}

\newcommand{\odgovor}[1]{\textcolor{blue}{#1}}

\begin{document}

\title{Automatizacija testiranja softvera\\ \small{Milos Manić, Dalibor Stanisavljević}}

\maketitle
\tableofcontents

\chapter{Uputstva}
\emph{Prilikom redavanja odgovora na recenziju, obrišite ovo poglavlje.}

Neophodno je odgovoriti na sve zamerke koje su navedene u okviru recenzija. Svaki odgovor pišete u okviru okruženja \verb"\odgovor", \odgovor{kako bi vaši odgovori bili lakše uočljivi.} 
\begin{enumerate}

\item Odgovor treba da sadrži na koji način ste izmenili rad da bi adresirali problem koji je recenzent naveo. Na primer, to može biti neka dodata rečenica ili dodat pasus. Ukoliko je u pitanju kraći tekst onda ga možete navesti direktno u ovom dokumentu, ukoliko je u pitanju duži tekst, onda navedete samo na kojoj strani i gde tačno se taj novi tekst nalazi. Ukoliko je izmenjeno ime nekog poglavlja, navedite na koji način je izmenjeno, i slično, u zavisnosti od izmena koje ste napravili. 

\item Ukoliko ništa niste izmenili povodom neke zamerke, detaljno obrazložite zašto zahtev recenzenta nije uvažen.

\item Ukoliko ste napravili i neke izmene koje recenzenti nisu tražili, njih navedite u poslednjem poglavlju tj u poglavlju Dodatne izmene.
\end{enumerate}

Za svakog recenzenta dodajte ocenu od 1 do 5 koja označava koliko vam je recenzija bila korisna, odnosno koliko vam je pomogla da unapredite rad. Ocena 1 označava da vam recenzija nije bila korisna, ocena 5 označava da vam je recenzija bila veoma korisna. 

NAPOMENA: Recenzije ce biti ocenjene nezavisno od vaših ocena. Na osnovu recenzije ja znam da li je ona korisna ili ne, pa na taj način vama idu negativni poeni ukoliko kažete da je korisno nešto što nije korisno. Vašim kolegama šteti da kažete da im je recenzija korisna jer će misliti da su je dobro uradili, iako to zapravo nisu. Isto važi i na drugu stranu, tj nemojte reći da nije korisno ono što jeste korisno. Prema tome, trudite se da budete objektivni. 

\chapter{Prvi recenzent \odgovor{--- ocena:5} }
\section{O čemu rad govori?}
Treba težiti automatskom testiranju jer štedi vreme i sprečava programera da pravi greske čak i u ranim fazama programiranja. Testovi pomažu programeru da se fokusira na funkcionalnost koda. Testiranje se deli na statičko i dinamičko, a dinamičko se dalje deli na ne funkcionalno i funkcionalno.

\section{Krupne primedbe i sugestije}
Strukturno testiranje nije opisano kao deo dinamičkog testiranja (nije sinonim za ne funkcionalno).
\odgovor{Iz literature koju smo uspeli da nađemo, strukturno testiranje je deo druge podele koju mi nismo naveli u našem radu, što zbog odrednica o broju strana, što zbog toga što je manje bitna u odnosu na ostale podele u odnosu na automatsko testiranje. Ta podela je podela u odnosu na poznavanje koda sistema koji se testira, tj "box" podela. Strukturno testiranje bi bilo neki sinonim White-box testiranja, u odnosu na Black-box testiranje. Podele su uglavnom potpuno nezavisne, tako da testiranje može da bude funkcionalno i strukturno, ili nefunkcionalno i strukturno i obrnuto itd.}
\section{Sitne primedbe}
Sintaksne greške.
\odgovor{Sve sintaksne greške su ispravljene.}

\section{Provera sadržajnosti i forme seminarskog rada}
% Oдговорите на следећа питања --- уз сваки одговор дати и образложење

\begin{enumerate}
\item Da li rad dobro odgovara na zadatu temu?\\
Rad je dobro odgovorio na zadatu temu, ako zanemarimo prethodno navedene primedbe, jer sadrži osnovne podele i pojmove o automatskom testiranju softvera, kao i određene primere.

\item Da li je nešto važno propušteno?\\
Nedostaje strukturno testiranje, a nista nije spomenuto ni u vezi sa metodologijama za automatizaciju testiranja softvera.
\odgovor{Za strukturno odgovoreno ranije, metodologije su u poglavlju 7 pod nazivom pristupi automatizaciji. Zbog zahteva o obimu rada, nismo ulazili u previše detalja}
\item Da li ima suštinskih grešaka i propusta?\\
Nisam uočio dodatne velike greške u radu osim onih koje su prethodno navedene.

\item Da li je naslov rada dobro izabran?\\
Mislim da bi "Osnove automatskog testiranja softvera" bilo adekvatno.
\odgovor{Promenjeno}
\item Da li sažetak sadrži prave podatke o radu?\\
Sažetak više liči uvod nego na opis sadržaja rada. 
\odgovor{Promenjen sažetak}
\item Da li je rad lak-težak za čitanje?\\
Rad je lak za čitanje.

\item Da li je za razumevanje teksta potrebno predznanje i u kolikoj meri?\\
Potrebno je poznavati osnove jezika c++ da bi se razumeo primer sa CppUnit kodom.

\item Da li je u radu navedena odgovarajuća literatura?\\
Literatura je u redu ako zanemarimo korektnost wikipedije.
\odgovor{Wikipedia je korišćena samo kao pomoć pri nabrajanju nefunkcionalnih tehnika testiranja, i kao neki zbirni dokument većine tehnika. Dodata još jedna referenca na literaturu}

\item Da li su u radu reference korektno navedene?\\
Reference su u redu, ali postoje delovi koji nemaju referencu.
\odgovor{Dodate su reference koje nedostaju.}
\item Da li je struktura rada adekvatna?\\
Struktura je u redu.

\item Da li rad sadrži sve elemente propisane uslovom seminarskog rada (slike, tabele, broj strana...)?\\
Što se tiče uslova rad je u redu.

\item Da li su slike i tabele funkcionalne i adekvatne?\\
Slike i tabela su adekvatne.

\end{enumerate}

\section{Ocenite sebe}
% Napišite koliko ste upućeni u oblast koju recenzirate: 
% a) ekspert u datoj oblasti
% b) veoma upućeni u oblast
 c) srednje upućeni
% d) malo upućeni 
% e) skoro neupućeni
% f) potpuno neupućeni
% Obrazložite svoju odluku

Znanje iz kursa razvoj softvera.

Čitao sam dodatnu literaturu da bi bolje recenzirao rad.



\chapter{Drugi recenzent \odgovor{--- ocena:5} }

\section{О чему рад говори?}
% Напишете један кратак пасус у којим ћете својим речима препричати суштину рада (и тиме показати да сте рад пажљиво прочитали и разумели). Обим од 200 до 400 карактера.
Рад говори о аутоматском тестирању софтвера. Састоји се из три целине, и то су статичко и динамичко тестирање, и грануларност тестирања. Описани су приступи тестирању софтвера и наведен је пример употребе CPPUnit-a. Такође се истичу и предности и мане оваквог приступа развоју софтвера. 

\section{Крупне примедбе и сугестије}
% Напишете своја запажања и конструктивне идеје шта у раду недостаје и шта би требало да се промени-измени-дода-одузме да би рад био квалитетнији.
Немам крупних примедби. Рад делује као целина. 

\section{Ситне примедбе}
% Напишете своја запажања на тему штампарских-стилских-језичких грешки
Рад је писан латиницом. Негде су остављени енглески термини.
\odgovor{Latinica bi trebala da je u redu što se tiče zahteva rada, engleski termini su prevedeni.}

\section{Провера садржајности и форме семинарског рада}
% Oдговорите на следећа питања --- уз сваки одговор дати и образложење

\begin{enumerate}
\item Да ли рад добро одговара на задату тему?\\
Мислим да рад објашњава шта је аутоматско тестирање и да кроз практичне примере то и демонстрира, па самим тим и да одговара теми.
\item Да ли је нешто важно пропуштено?\\
Не познајем довољно добро тему, али на мене рад оставља утисак да није.
\item Да ли има суштинских грешака и пропуста?\\
Мислим да нема.
\item Да ли је наслов рада добро изабран?\\
Рад се бави аутоматским тестирањем софтвера и то је у наслову и наглашено, па је он прикладан.
\item Да ли сажетак садржи праве податке о раду?\\
Сажетак мало говори о самом раду, више о теми, тако да би требало написати шта се конкретно из ове области налази у овом раду.
\odgovor{Promenjen sažetak}
\item Да ли је рад лак-тежак за читање?\\
Рад може да делује мало напорнијим за читање.
\item Да ли је за разумевање текста потребно предзнање и у коликој мери?\\
Потребно је предзање. Нпр. код примера у CPPUnit-у је неопходно претходно познавање програмског језика C++.
\item Да ли је у раду наведена одговарајућа литература?\\
Наведена литература је одговарајућа.
\item Да ли су у раду референце коректно наведене?\\
Референце су исправно наведене.
\item Да ли је структура рада адекватна?\\
Рад има добру структуру.
\item Да ли рад садржи све елементе прописане условом семинарског рада (слике, табеле, број страна...)?\\
Рад садржи све захтеване елементе.
\item Да ли су слике и табеле функционалне и адекватне?\\
Слике су доста илустративне, док је табела могла да буде мало лепше форматирана (Могло је да буде наглашено заглавље табеле).
\odgovor{Naglašeno zaglavlje tabele}
\end{enumerate}

\section{Оцените себе}
% Napišite koliko ste upućeni u oblast koju recenzirate: 
% a) ekspert u datoj oblasti
% b) veoma upućeni u oblast
% c) srednje upućeni
% d) malo upućeni 
% e) skoro neupućeni
% f) potpuno neupućeni
% Obrazložite svoju odluku
Скоро сам неупућен у ову област. До сада се нисам бавио аутоматским тестирањем.


\chapter{Treći recenzent \odgovor{--- ocena:5} }

\section{O čemu rad govori?}
% Напишете један кратак пасус у којим ћете својим речима препричати суштину рада (и тиме показати да сте рад пажљиво прочитали и разумели). Обим од 200 до 400 карактера.
U ovom radu govori se o procesu automatskog testiranja, zbog čega je ono važno pri razvoju softvera, a takođe se osvrće i na to koji se problemi mogu javiti pri ovom procesu. U radu su navedene pojedine prednosti automatizacije procesa testiranja, a prikazani su i konkretni pristupi automaizaciji, kroz objašnjenje tri generacije sistema za testiranje. Objašnjene su i razlike između statičkog i dinamičkog testiranja prikazivanjem nekih konkretnih tehnika. U sklopu dinamičkog testiranja, objašnjene su razlike između testiranja funkcionalnih i nefunkcionalnih zahteva. Za nefunkcionalne zahteve pružen je skup tehnika koje se odnose na različite probleme koji se mogu javiti u softveru, dok su za funkcionalne zahteve predstavljeni uobičajeni koraci testiranja. Poseban akcenat stavljen je na granularnost jedinica koje se testiraju, čime je prikazano šta sve u sistemu treba testirati, a time i koja su najčešća mesta na kojima se mogu javiti greške pri razvoju softvera. Ovde su objašnjeni pojmovi kao što su \textit{testovi jedinica koda}, \textit{testovi integracije} i \textit{testovi sistema}. Pružen je takođe i praktičan primer testiranja jedinice koda korišćenjem biblioteke \textit{CppUnit} i objašnjen je način korišćenja ove biblioteke. Objašnjeno je i zbog čega je testiranje grafičkog interfejsa bolje automatizovati nego raditi ručno, a navedeni su i načini na koje se mogu predstaviti konačni automati za testiranje grafičkog interfejsa. 

\section{Krupne primedbe i sugestije}
% Напишете своја запажања и конструктивне идеје шта у раду недостаје и шта би требало да се промени-измени-дода-одузме да би рад био квалитетнији.
Nemam krupnijih primedbi niti sugestija, rad je odgovarajući u svom trenutnom obliku.

\section{Sitne primedbe}
% Напишете своја запажања на тему штампарских-стилских-језичких грешки
Povremeno izostavljanje dijakritičkih znakova za srpska slova, pogotovo u prvom delu rada. Nemam drugih primedbi osim ove.
\odgovor{Ispravljeno}

\section{Provera sadržajnosti i forme seminarskog rada}
% Oдговорите на следећа питања --- уз сваки одговор дати и образложење

\begin{enumerate}
\item Da li rad dobro odgovara na zadatu temu?\\

	Rad u potpunosti odgovara na zadatu temu.
\item Da li je nešto važno propušteno?\\

	S obzirom na moje znanje o temi, nisam kompetentan da procenim da li je izostavljeno nešto važno. Rad je dobro zaokružen, tako da mislim da nije izostavljeno nešto što bi trebalo biti u radu.
\item Da li ima suštinskih grešaka i propusta?\\

	Nema suštinskih grešaka niti propusta.
\item Da li je naslov rada dobro izabran?\\

	Naslov rada sasvim odgovara sadržaju rada.
\item Da li sažetak sadrži prave podatke o radu?\\

	Sažetak dobro opisuje značaj i mesto automatizacije procesa testiranja, eventualno bi trebao da malo bolje upozna čitaoca sa konkretnim sadržajem rada, tj. sa obradjenim temama.
\odgovor{Sažetak ispravljen}
\item Da li je rad lak-težak za čitanje?\\

	Rad je lak za čitanje. Jedina preporuka koju bih imao jeste korišćenje paketa \textit{Listings} za prikaz kodova, što bi dodatno poboljšalo čitljivost.
\odgovor{Korišćen je paket \textit{Listings}.}
\item Da li je za razumevanje teksta potrebno predznanje i u kolikoj meri?\\

	Nije potrebno predznanje da bi se rad razumeo. Svaki korišćeni termin je prethodno objašnjen.
\item Da li je u radu navedena odgovarajuća literatura?\\

	Navedena je sva odgovarajuća literatura. Ipak, kao jedan od izvora navedena je i "Wikipedia", koju ne bi trebalo uzimati kao pouzdan izvor.
\odgovor{Dodato još referenci}
\item Da li su u radu reference korektno navedene?\\

	Reference su korektno navedene. Ipak, nisam bio u prilici da proverim reference koje se odnose na izvor "L.G. Hayes. The Automated Testing Handbook. Software Testing In-
	stitute, 2004.", pošto nemam pomenutu knjigu.
\odgovor{\url{http://www.softwaretestpro.com/itemassets/4772/automatedtestinghandbook.pdf} P. S. Nadam se da nije piratska verzija, pošto je sa legitimnog sajta, ali nije sigurno, pa nema tog linka u radu}
\item Da li je struktura rada adekvatna?\\

	Struktura rada je adekvatna, rad je podeljen na odgovarajuće celine.
\item Da li rad sadrži sve elemente propisane uslovom seminarskog rada (slike, tabele, broj strana...)?\\

	Rad sadrži sve elemente propisane uslovom seminarskog rada (9 strana, 2 slike, tabela, sažetak, uvod, zaključak, bibliografija)
\item Da li su slike i tabele funkcionalne i adekvatne?\\

	Iako je iz konteksta jasno na šta se slike i tabele odnose, na njih se nigde u tekstu ne referencira eksplicitno. Ako se to zanemari, slike i tabele su funkcionalne i adekvatne. 
\odgovor{Dodato je referenciranje na slike i tabele.}
\end{enumerate}

\section{Ocenite sebe}
% Napišite koliko ste upućeni u oblast koju recenzirate: 
% a) ekspert u datoj oblasti
% b) veoma upućeni u oblast
% c) srednje upućeni
% d) malo upućeni 
% e) skoro neupućeni
% f) potpuno neupućeni
% Obrazložite svoju odluku
U ovu temu sam srednje upućen. Sa ovim problemom sretao sam se u okviru kursa \textit{Razvoj softvera}, a koristio sam u ličnom radu biblioteku \textit{CppTest Unit}.



\chapter{Četvrti recenzent \odgovor{--- ocena:5} }

\section{O čemu rad govori?}
% Напишете један кратак пасус у којим ћете својим речима препричати суштину рада (и тиме показати да сте рад пажљиво прочитали и разумели). Обим од 200 до 400 карактера.
Testiranje softvera je deo razvoja softvera koji programeri, svesno ili nesvesno, primenjuju. Neka testiranja su moguća bez pokretanja programa (ispravljanje sintaksnih grešaka), dok je za neka druga neophodno da se program pokrene. Automatsko testiranje je jako poželjno, a u nekim situacijama i neophodno. Postoje brojne prednosti u odnosu na ručno testiranje (ušteda vremena, češće testiranje...).
\\

\section{Krupne primedbe i sugestije}
% Напишете своја запажања и конструктивне идеје шта у раду недостаје и шта би требало да се промени-измени-дода-одузме да би рад био квалитетнији.
Veliki deo rada je napisan 'osisanom' latinicom, što se mora ispraviti.\\
\odgovor{Poslata je prethodna verzija rada umesto najnovije. Ispravljeno}
Sažetak bi bolje izgledao da se postojeća rečenica zameni rečenicom: 
"Automatsko testiranje, kao što je statička analiza koda koja se izvršava u razvojnom okruženju, vrlo često programeri ni ne primećuju."\\
\odgovor{Prepravljeno}
\textbf{Deo 2:}\\
Prva rečenica je definicija (najverovatnije je prevedena od reči do reči) i mora biti okružena znacima navoda.
\odgovor{Nije prevod definicije, mada postoje slične rečenice u literaturi koju smo koristili. Rečenica predstavlja čisto opis metoda, i zaključak je iz referencirane literature}
Kada se navode reference, kao npr. kod tehnika statičkog testiranja, nema potrebe da referenca stoji nakon svake stavke podele. Navođenje referenci bi moglo da izgleda: \\
"Navedene su tehnike statičkog testiranja od najneformalnije do najformalnije (i jedine pogodne za automatizaciju) [3]:" i nakon ovoga sledi podela. Ovo se javlja na više mesta u daljem tekstu.\\
\odgovor{Promenjeno: u delu 2, 3.1 i 7}\\
\textbf{Deo 4.2:}\\
Primeri koda treba da se ubace kao slike i u radu se treba pozvati na njih. \\
\odgovor{Korišćen je paket \textit{Listings} za prikazivanje koda.}
\textbf{Deo 4.3:}\\
Konstruisanje konačnog automata za testiranje grafičkog interfejsa bi moglo da se stavi u podsekciju 4.3.1.\\
\odgovor{Urađeno}


\section{Sitne primedbe}
% Напишете своја запажања на тему штампарских-стилских-језичких грешки
U radu, na više mesta nakon znaka ',' stoji veznik 'i' što je u srpskom jeziku suvišno. Koristi se ili znak ',' ili veznik 'i'. Nakon znaka ',' mora da sledi blanko karakter\odgovor{Ispravljeno}. Ako se koriste zagrade, pre '(' treba da stoji blanko karakter.\odgovor{Ispravljeno} 


\section{Provera sadržajnosti i forme seminarskog rada}
% Oдговорите на следећа питања --- уз сваки одговор дати и образложење

\begin{enumerate}
\item Da li rad dobro odgovara na zadatu temu?\\
Da. U radu je opisana automatizacija testiranja.\\
\item Da li je nešto važno propušteno?\\
Deluje da nije.\\
\item Da li ima suštinskih grešaka i propusta?\\
Deluje da nema.\\
\item Da li je naslov rada dobro izabran?\\
Da. Naslov rada i sadržaj rada se poklapaju.\\
\item Da li sažetak sadrži prave podatke o radu?\\
Da. Sažetak sadrži suštinu rada.\\
\item Da li je rad lak-težak za čitanje?\\
Zavisi od sekcije do sekcije. Uvod je nezgodan za čitanje, najviše zbog osisane latinice, u delu 4.2 pasus pre prve slike je nejasan.\odgovor{Ispravljen je problem sa latinicom.Smatram da je pasus pre prve slike jasan.} Ostatak rada je manje-više lak za čitanje.\\
\item Da li je za razumevanje teksta potrebno predznanje i u kolikoj meri?\\
Potrebno je predznanje iz oblasti informatike radi razumevanja nekih "osnovnih" pojmova.\\
\item Da li je u radu navedena odgovarajuća literatura?\\
Da.\\
\item Da li su u radu reference korektno navedene?\\
Autori rada se pozivaju na sve reference. U sugestijama je navedeno šta bi trebalo izmeniti.\\
\item Da li je struktura rada adekvatna?\\
Da.\\
\item Da li rad sadrži sve elemente propisane uslovom seminarskog rada (slike, tabele, broj strana...)?\\
Rad sadrži sve osim ključnih reči.\\
\odgovor{Dodat je taj deo.}
\item Da li su slike i tabele funkcionalne i adekvatne?\\
Ne. U delu 4.2 primeri su u formi slike i nedostaje im naslov. \odgovor{Ispravljeno.}U delu 4.3 slike nemaju naslov.\odgovor{Ispravljeno.} Druga slika u delu 4.3 je nejasna.\odgovor{Mislim da je slika sasvim jasna.} U radu nema pozivanja na slike. \odgovor{Ispravljeno.}Tabela je adekvatno odrađena i razumljiva. Iako se autori pozivaju na tablu u Keyword-driven pristupu, ne pozivaju se pravilno.\odgovor{Prepravljeno} \\

\end{enumerate}

\section{Ocenite sebe}
% Napišite koliko ste upućeni u oblast koju recenzirate:
% a) ekspert u datoj oblasti
% b) veoma upućeni u oblast
% c) srednje upućeni
% d) malo upućeni
% e) skoro neupućeni
% f) potpuno neupućeni
% Obrazložite svoju odluku
d) Neki delovi rada su pominjani na predmetu Razvoj softvera koji drži profesor Saša Malkov.\\


\chapter{Dodatne izmene}
%Ovde navedite ukoliko ima izmena koje ste uradili a koje vam recenzenti nisu tražili. 

\end{document}



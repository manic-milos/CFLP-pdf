% !TEX encoding = UTF-8 Unicode

\documentclass[a4paper]{article}

\usepackage{color}
\usepackage{url}
\usepackage[T2A]{fontenc} % enable Cyrillic fonts
\usepackage[utf8]{inputenc} % make weird characters work
\usepackage{graphicx}
\usepackage{amsmath}
\usepackage{listings}
\lstset{language=C++,
    frame=tb, % draw a frame at the top and bottom of the code block
    tabsize=4, % tab space width
    showstringspaces=false, % don't mark spaces in strings
    numbers=left, % display line numbers on the left
    commentstyle=\color{green}, % comment color
    keywordstyle=\color{blue}, % keyword color
    stringstyle=\color{red} % string color
}

\usepackage[english, serbian]{babel}
%\usepackage[english, serbianc]{babel} %ukljuciti babel sa ovim opcijama, umesto gornjim, ukoliko se koristi cirilica

\usepackage[unicode]{hyperref}
\hypersetup{colorlinks, citecolor=green, filecolor=green, linkcolor=blue, urlcolor=blue}

%\newtheorem{primer}{Пример}[section] %ćirilični primer
\newtheorem{primer}{Primer}[section]

\begin{document}

\title{Hard capacitated k-facility location problem \\ \small{Seminarski rad u okviru kursa\\
Matematičko programiranje i optimizacija\\ Matematički fakultet}}

\author{
Student:Miloš Manić 1087/2014\\
Problem:14\\
Metode: Genetski algoritmi, iterativna lokalna pretraga i njihova hibridizacija}
\date{9.~april 2015.}
\maketitle

\abstract{


}


\tableofcontents

\newpage

\section{Problem}
\subsection{Matematička formulacija problema}
Capacitated k-facility location problem(CKFL) se može formulisati kao sledeći Mixed Integer Problem(MIP) gde promenljiva $x_{ij}$ označava količinu potražnje klijenta $j$ koja je opslužena postrojenjem $i$, a $y_i$ označava da li je postrojenje $i$ otvoreno:

\begin{align}
min \sum_{i \in F}\sum_{j \in D}c_{ij}x_{ij} + \sum_{j \in F}f_iy_i\\
 subject\: to: 
\quad \sum_{i \in F}x_{ij} = d_j, \forall j \in D,\quad\\
 \sum_{j \in D}x_{ij} \le s_iy_i, \forall i \in F,\\
\sum_{i \in F}y_i \le k,\\
x_{ij} \ge 0, \forall i \in F,\forall j \in D,\\
y \in \{ 0 , 1 \}, \forall i \in F
\end{align}
\subsection{Opis problema}

U datom probllemu dat je skup klijenata $D$ i skup potencijalnih postrojenja(lokacija na kojima se može izgraditi postrojenje $F$.
\begin{itemize}
\item Svako postrojenje $i \in F$ ima kapacitet $s_i$
\item Izgradnja postrojenja $i \in F$ košta $f_i$
\item Svaki klijent $j \in D$ ima potražnju $d_i$
\item Slanje $x_{ij}$ jedinica robe od postrojenja $i$ do klijenta $j$ košta $c_{ij}x_{ij}$, gde je $c_{ij}$ jedinicna cena proporcionalna rastojanju između $i$ i $j$
\item Na svakoj potencijalnoj lokaciji $i \in F$ može se izgraditi najviše jedno postrojenje
\item Bez gubitka opštosti može se smatrati da su cene izgradnje $f_i$, kapaciteti $s_i$, i potražnje $d_j$ celi brojevi

Cilj je opslužiti sve klijente koristeći najviše $k$ postrojenja sa što manjim(minimalnim) troškovima izgradnje postrojenja i dopremanja robe.
\end{itemize}

\subsection{Primena}

\subsection{Postojeći načini rešavanja}




\addcontentsline{toc}{section}{Literatura}
\appendix
\bibliography{seminarski} 
\bibliographystyle{plain}


\end{document}
